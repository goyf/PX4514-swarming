% ****** Start of file apssamp.tex ******
%
%   This file is part of the APS files in the REVTeX 4.1 distribution.
%   Version 4.1r of REVTeX, August 2010
%
%   Copyright (c) 2009, 2010 The American Physical Society.
%
%   See the REVTeX 4 README file for restrictions and more information.
%
% TeX'ing this file requires that you have AMS-LaTeX 2.0 installed
% as well as the rest of the prerequisites for REVTeX 4.1
%
% See the REVTeX 4 README file
% It also requires running BibTeX. The commands are as follows:
%
%  1)  latex apssamp.tex
%  2)  bibtex apssamp
%  3)  latex apssamp.tex
%  4)  latex apssamp.tex

\documentclass[
reprint,
%preprint,
%superscriptaddress,
%groupedaddress,
%unsortedaddress,
%runinaddress,
%frontmatterverbose,
showpacs,
preprintnumbers,
%nofootinbib,
%nobibnotes,
bibnotes,
amsmath,
amssymb,
aps,
pra,
%prl,
%prb,
%rmp,
%prstab,
%prstper,
floatfix,
]{revtex4-1}

\usepackage{graphicx} % Include figure files
\usepackage{dcolumn} % Align table columns on decimal point
\usepackage{bm} % bold math
\usepackage{hyperref} % add hypertext capabilities
%\usepackage[mathlines]{lineno}% Enable numbering of text and display math
%\linenumbers\relax % Commence numbering lines

%\usepackage[showframe,%Uncomment any one of the following lines to test
%%scale=0.7, marginratio={1:1, 2:3}, ignoreall,% default settings
%%text={7in,10in},centering,
%%margin=1.5in,
%%total={6.5in,8.75in}, top=1.2in, left=0.9in, includefoot,
%%height=10in,a5paper,hmargin={3cm,0.8in},
%]{geometry}

\begin{document}

\title{Animal Swarming} % Force line breaks with \\
\thanks{Simulation \& Theory}

\author{Jan Polivka (ID:51553399)}
 \email{j.polivka.15@aberdeen.ac.uk}
\author{Struan Murray (ID:51550442)}
 \email{struan.murray.15@aberdeen.ac.uk}
\author{Daniel Bain (ID:51443692)}
 \email{daniel.m.bain.14@aberdeen.ac.uk}
\affiliation{University of Aberdeen}

\date{\today}

\begin{abstract}
	\noindent{The swarming of animals has been known to form complex patterns and this report covers the basic processes required to recreate these forms in a simulation as well as the theory and analysis of these swarms as a form of defence against predators.}
\end{abstract}

\pacs{\href{https://ufn.ru/en/pacs/87.19.St/}{87.19.St}, \href{https://ufn.ru/en/pacs/all}{87.19.lp}}
\maketitle

%\tableofcontents

\section{\label{sec:introduction}Introduction to Swarming}

The collective movement of animals can produce eye-catching patterns in a seemingly complex yet ordered fashion. (FIG. \ref{fig:birdswarmphoto})
These patterns can hypothetically be assigned to the similar simultaneous reactions of the individual animals to a danger; the prospect of food or safety.
Similar patterns of ``Swarming" have also recently been discovered in purely chemical reactions where chemical units are self-propelled.\cite{collectivemotion}
This suggests that a large proportion of swarming behaviour occurs as a result of very simple rules of interaction between the animals.

\begin{figure}[!htp]
	\fbox{\includegraphics[width=0.9\linewidth]{images/james-wainscoat-swarm.jpg}}

	\caption{
		A classic example of swarming starlings.\cite{jameswainscoatphoto}
		Here the swarm has coincidently taken on a shape resembling a whale.
	}

	\label{fig:birdswarmphoto}
\end{figure}

\section{\label{sec:behaviours}Standard Behaviours}
A large proportion of swarming behaviour is because of the base instincts of fear and necessary social interaction.
The heavy emphasis of social interaction is a result of years of evolution requiring the animals to seek safety in numbers against their predators.
The various theories behind swarming range from complex ``group minds"\cite{diffusionadaption} to the appreciably simplified set of rules which govern what is effectively a stochastic model with few rules.\cite{modellingflocks}
These simplified behaviours were classically modelled in 1986 with a program named ``\textbf{Boids}"\cite{boids} to great success.
Therefore programmed individual animals may be referred to as ``Boids" - Short for "bird-oid object" - though the term may also be applied to fish.
This program provides a set of rules to allow the simple simulation of swarming animals.

\subsection{\label{sec:boidrejection}Reaction to a predator}

\begin{figure}[!htp]
	\includegraphics[width=0.65\linewidth]{images/predator.png}

	\caption{
		Several animals (black) fly from a predator (red).
		The velocity at which the animal attempts to escape is directly related to the relative location of the predator.
	}

	\label{fig:animalpredator}
\end{figure}

The most drastic change to swarm behaviour occurs when a predator, lapse in cohesion or any other danger to the swarm occurs.
The natural instinct for any swarm-capable animal, when faced with immediate danger, is to quickly remove itself from the general path of said danger.
This results in a swarm reacting - and possibly collapsing - in one of the following scenarios:
\begin{description}
    \item{An individual animal leaves the swarm - The tamest scenario for the swarm, an animal is expelled and made vulnerable but the swarm retains integrity.}
    \item{The swarm splits - If there is a large enough disturbance, but animal alignment remains in synch, the swarm may split into separate smaller swarms which may merge at a later time.}
    \item{Chaotic collapse - If animal alignment in the swarm is challenged (i.e. a sufficiently large group within the swarm rapidly changes direction) then the swarm may rapidly expand and lose its protective properties (FIG. \ref{fig:animalpredator}), requiring complete reorganisation.}
\end{description}

\subsection{\label{sec:boidattraction}Reaction to a friend}

\begin{figure}[!htp]
		\fbox{\includegraphics[width=0.45\linewidth]{images/separation.png}}
		\label{fig:boidseparation}
		\fbox{\includegraphics[width=0.45\linewidth]{images/cohesion.png}}
		\label{fig:boidcohesion}
		\fbox{\includegraphics[width=0.45\linewidth]{images/alignment.png}}
		\label{fig:boidalignment}

	\caption{The main principles of direct flock interaction: Separation, Cohesion, Alignment.\cite{boids}}

	\label{fig:flockinteraction}
\end{figure}

The integrity of a swarm relies on the animals being aware of their surroundings, speed and direction.
This is almost entirely governed by neighbouring animals in a swarm.
This is one of the sticking points in a model not based on OOP principles.
Each animal or represented point has to act independently of the other points and calculation complexity increases to the power of points simulated.
To support stability, each animal in a swarm must maintain some distance from the animals surrounding it.
The animals, however, must be as closely packed as possible.
The information that is carried through the swarm starts out slowly but quickly results in a rapid change to the swarm, similar to the butterfly effect. This has been observed to be somewhat exponential.
As a result of these rules, the creatures tend to align themselves in a similar way to their neighbours.
These calculations are carried out for each animal and in relation to every other animal.

\subsection{\label{sec:leaderattraction}Reaction to a goal/leader}

\begin{figure}[!htp]
	\includegraphics[width=0.65\linewidth]{images/leader.png}

	\caption{
		Several animals (in black) fly towards a goal (in blue).
		The velocity at which the animal attempts to reach the goal is positively affected by distance.
	}

	\label{fig:boidleader}
\end{figure}

To generally control the swarm, a target (for example a food source) must be utilised in the model.
In the simplest case, this may be the centre of the screen (0,0).
It may also be a moveable point, possibly representing a desired direction or leader, although studies have shown that swarming lacks a definite leader in almost all cases.\cite{modellingflocks}
The draw to the leader is programmatically defined to be directly proportional to the distance from the target. (FIG. \ref{fig:boidleader})

\subsection{\label{sec:birdsvfish}Combined behaviour of birds and fish}
Animals swarm for the purpose of safety.
Birds and fish are a good example of this with both having different reasons and protection mechanisms to enable this.\\
\indent Birds have extremely keen eyesight and hiding in large numbers does not work.
Because of gravity, they must must actively engage in flight motion, which requires the target to have enough space around it, and so predators are forced to hunt birds that are at the edge of the flock, which in turn incentivises cohesion among flocking birds.\\
\indent Bird flocking is a relatively simple behaviour that does not have any specific shape or form, disregarding a flock of migratory birds.
Their defence against a predator is also quite primitive but actually very simple.
When attacked, the birds simply disperse into all directions. This is effective because the actual predator has to avoid a collision as well and has to be extremely mindful of what it is attacking and focus only at the edges of the flock.\\
\indent Conversely, fish swarm because the aquatic environment scatters and absorbs light, which gives the fish somewhat of a cover and therefore as the predator randomly searches the environment, they are less likely to find the fish. (FIG. \ref{fig:swarmblindness})
However, the fish must also have a mechanism of how to avoid being eaten.
As mentioned before, fish need to have a mechanism to avoid being eaten when faced with a predator.
Schools have adopted the so-called fountain manoeuvre (FIG. \ref{fig:fountain}), which takes advantage of the predator's momentum by swimming towards it when it attacks.

\begin{figure}[!htp]
	\includegraphics[width=1.0\linewidth]{images/partridgeFountain.png}

	\caption{
		The fountain manoeuvre step by step.
	}

	\label{fig:fountain}
\end{figure}

\section{\label{sec:oop}Models \& Object Oriented Programming (OOP)}

Mathematica does not support object-oriented programming, so the majority of calculations for swarms must be solved using matrices.
The crux of the matter is that the swarm, leader and predator all behave differently, so must be programmed as separately as possible.
The 3 differently behaving types should also be able to interact and obtain data from each other's positions.

\subsection{\label{sec:birdswarm}1st Model - Bird Swarm}

\begin{figure}[!htp]
	\fbox{\includegraphics[width=0.7\linewidth]{images/lines.png}}

	\caption{
		Animals form lines around the leader.
		The predator has succeeded in separating a group from the main swarm.
	}

	\label{fig:animallines}
\end{figure}

The first attempted solution involved the strict separation of:
\begin{description}
	\item{Swarming Animals - Known as boids.}
	\item{Leader - To allow a target point, or food source.}
	\item{Predator - To disturb the group of boids.}
\end{description}

The birds were programmed to randomly choose another bird to follow, and a bird to avoid.
Each calculation round would have the bird move towards the followed animal based on the Euclidean distance between them and away from the avoided animal based on the inverse.

The leader and predator were programmed to follow intersecting paths to simulate the collision of the hunter and the swarm.
In both cases, the speed was normalised to be independent of movement path diameter.

The program initially assigns the boids to a random point on a map of fixed size.
In each calculation stage, a unit vector, representing the direction of the leader to the boid is calculated.
This unit vector is multiplied by a scalar value corresponding to a fraction of the distance between the boid and the leader.
The resulting vector is added to each boids current position.

A nearly identical process occurs for the enemy, however, it moves the boid away from the predator.
The scalar value also increases with the decrease in Euclidean distance, resulting in a large reaction should the predator be nearby.

To calculate the movement between boids, a matrix containing a list of followers calculates a new matrix based on the distance between each boid and the boid it is following.
This vector distance is then divided by a scalar value and added to each boid.
The same process occurs in reverse for each boids enemy.
The working code is shown in \href{https://youtu.be/EyhdRhi5Gnc}{2D Video} and \href{https://youtu.be/KG7qQum6EUc}{3D Video} formats.

\subsection{\label{sec:fishswarm}2nd Model - Fish Swarm}

\begin{figure}[]
		\centering
		\fbox{\includegraphics[width=0.7\linewidth]{images/Fish_Swarm.png}}
		\fbox{\includegraphics[width=0.7\linewidth]{images/Fish_NoSwarm.png}}

	\caption{The effects of swarming mitigating the ability (represented by the black circle) of the hunter to detect and predate on the fish.\cite{underwatervision}}

	\label{fig:swarmblindness}
\end{figure}

It was decided that a swarm did not inherently require a moveable target and instead should simply attempt to retain integrity.
To introduce more variables to the simulation, it was decided to move to a fish based simulation.
It is thought that of the 20,000+ known species of fish, half will at some point in there lives form a school.\cite{fishschools}
This led to the development of a second model, loosely based on the first.
The behaviour of the predator was also improved to instead pick a specific fish and hunt it down.
The interactions also allow the predator to ``eat" the fish, removing it from the swarm.

One of the most distinctive properties of an underwater swarm, compared to anything above ground, is its use in evading enemies.
This is due to the ``blindness" of the predator and predated fish in murky waters.
The maximum theoretical distance visible underwater is 80 m, and the furthest observed in calm, clear water to date is 74 m.\cite{underwatervision}
By forming a tight swarm, the predator has a significantly reduced chance of randomly finding the fish than if they were spread throughout an area. (FIG. \ref{fig:swarmblindness})

To keep the fish on screen, a slight move towards the center of the model each calculation round was added.
Other vector components included were random noise and a move towards the center of the shoal of fish around the animal being calculated.
It was decided to use a logarithmic weighting for these calculations, where the closest fish represented the largest influence and a calculation for these weightings was calculated as a constant array, allowing quick application and the ability of the model to run in real time.
To gain data on the effectiveness of this swarming and hiding, a ``kill" function was added to the hunter, where boids within a certain distance of the hunter would be eaten, and the data for this function recorded for each calculation step, resulting in a plot of living fish with time.

\begin{figure}[!htp]
		\centering
		\fbox{\includegraphics[width=0.9\linewidth]{images/Boid_Population.png}}
		

	\caption{Population of fish with each calculation, assuming swarming (in blue) and randomly placed fish (in red).}
	\label{fig:population}
\end{figure}


\section{\label{sec:discussion}Discussion}
The fish swarming model was a significant improvement over the bird swarming model, with greater realism because of the improved predator behaviour and proper animal grouping, rather than simple matrix based calculations.
A noted issue with the model was the possibility of an endless loop where the predator would continuously orbit a target if it started towards the target at a low angle. (FIG. \ref{fig:popdecline})
To observe the swarm, and prevent it from leaving the screen, a small step towards origin was included. This is not necessary in real-life however and could be considered a flaw in the program.
Development of the hunter AI is a significant area for improvement, and would likely resolve some of the issues observed.

\begin{figure}[!htp]
		\centering
		\fbox{\includegraphics[width=0.45\linewidth]{images/popDec2D.png}}
		\fbox{\includegraphics[width=0.47\linewidth]{images/popDec3D.png}}
		\fbox{\includegraphics[width=0.8\linewidth]{images/popDec3DOrbiting.png}}
		

	\caption{Population decrease rate of fish with each calculation, from top left: 2D population decline, 3D population decline, an example of orbiting a single target for 250 calculation steps.}
	\label{fig:popdecline}
\end{figure}

\section{\label{sec:conclusion}Conclusion}
It is fairly clear that swarming increases the chance of survival for animals (FIG. \ref{fig:population}), despite having no physical advantage over their predators.
The model which took advantage of Mathematica's scripting alongside a little bit of matrix manipulation produced the best balance of performance and realism.
Both models highlighted the stochastic nature of swarming, with no two simulations being the same.
With the factors included in the model, 2D swarms tended to form a donut-like shape, which suggests that such a formation holds some stability for large swarms.


% ****** Past here be the bibliography ******

	\bibliographystyle{ieeetr}
	\bibliography{references}

\end{document}

% ****** End of file main.tex ******
